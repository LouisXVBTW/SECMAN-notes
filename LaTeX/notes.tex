\documentclass[11pt]{article}
\usepackage[utf8]{inputenc}	% Para caracteres en español
\usepackage{amsmath,amsthm,amsfonts,amssymb,amscd}
\usepackage{multirow,booktabs}
\usepackage[table]{xcolor}
\usepackage{fullpage}
\usepackage{lastpage}
\usepackage{enumitem}
\usepackage{fancyhdr}
\usepackage{mathrsfs}
\usepackage{wrapfig}
\usepackage{setspace}
\usepackage{calc}
\usepackage{multicol}
\usepackage{cancel}
\usepackage[retainorgcmds]{IEEEtrantools}
\usepackage[margin=3cm]{geometry}
\usepackage{amsmath}
\newlength{\tabcont}
\setlength{\parindent}{0.0in}
\setlength{\parskip}{0.05in}
\usepackage{empheq}
\usepackage{framed}
\usepackage[most]{tcolorbox}
\usepackage{xcolor}
\colorlet{shadecolor}{orange!15}
\parindent 0in
\parskip 12pt
\geometry{margin=1in, headsep=0.25in}
\theoremstyle{definition}
\newtheorem{defn}{Definition}
\newtheorem{reg}{Rule}
\newtheorem{exer}{Exercise}
\newtheorem{note}{Note}
\begin{document}
\setcounter{section}{0}
\title{Making a placeholder}

\thispagestyle{empty}
%%%%%%%%%%%%%%%%%% TITLE PLACE
\begin{center}
{\LARGE \bf Title Placeholder}\\
{\large subtitle}\\
placeholder
\end{center}
\section{SECTION PLACEHOLDDER}
\subsection{SUBSECTION}
Non equidem invideo, miror magis posuere velit aliquet. Quisque placerat facilisis egestas cillum dolore. Curabitur blandit tempus ardua ridiculus sed magna. Contra legem facit qui id facit quod lex prohibet. Petierunt uti sibi concilium totius Galliae in diem certam indicere.\\Lorem ipsum dolor sit amet, consectetur adipisici elit, sed eiusmod tempor incidunt ut labore et dolore magna aliqua. Nec dubitamus multa iter quae et nos invenerat.

\subsection{The Tides}
\begin{shaded}
\textbf{The Tidal Force} \newline
this is the shaded area
\end{shaded}

\subsection{The Angular Velocity Vector}
The rest of the notes and the chapter will over reference frames that are rotating with respect to the inertial reference frame, so angular velocity has to be used. 
\begin{defn}
\textbf{Euler's Theorem} - The most general motion of any body relative to a fixed point \textit{O} is a rotation about some axis through \textit{O} To specify this rotation about a given point O, we only have to give the direction of the axis and the rate of rotation, or angular velocity $\omega$. Because this has a magnitude and direction, it is an obvious choice to write this rotation vector as $\omega$, the angular velocity vector. That is:
\begin{equation}
\omega = \omega\textbf{u}
\end{equation}
Where \textbf{u} is the unit vector
\end{defn}
\begin{shaded}
\textbf{Vector Velocity}\newline
The velocity at any point, \textit{P} (position, \textit{r}) is given by:
\begin{equation}
v = \omega\  x \ r
\end{equation}
\end{shaded}
\subsection*{Addition of Angular Velocities}


\end{document}